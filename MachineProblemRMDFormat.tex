\documentclass[]{article}
\usepackage{lmodern}
\usepackage{amssymb,amsmath}
\usepackage{ifxetex,ifluatex}
\usepackage{fixltx2e} % provides \textsubscript
\ifnum 0\ifxetex 1\fi\ifluatex 1\fi=0 % if pdftex
  \usepackage[T1]{fontenc}
  \usepackage[utf8]{inputenc}
\else % if luatex or xelatex
  \ifxetex
    \usepackage{mathspec}
  \else
    \usepackage{fontspec}
  \fi
  \defaultfontfeatures{Ligatures=TeX,Scale=MatchLowercase}
\fi
% use upquote if available, for straight quotes in verbatim environments
\IfFileExists{upquote.sty}{\usepackage{upquote}}{}
% use microtype if available
\IfFileExists{microtype.sty}{%
\usepackage{microtype}
\UseMicrotypeSet[protrusion]{basicmath} % disable protrusion for tt fonts
}{}
\usepackage[margin=1in]{geometry}
\usepackage{hyperref}
\hypersetup{unicode=true,
            pdftitle={MachineProblemRMD},
            pdfauthor={Helen Mary Labao-Barrameda},
            pdfborder={0 0 0},
            breaklinks=true}
\urlstyle{same}  % don't use monospace font for urls
\usepackage{color}
\usepackage{fancyvrb}
\newcommand{\VerbBar}{|}
\newcommand{\VERB}{\Verb[commandchars=\\\{\}]}
\DefineVerbatimEnvironment{Highlighting}{Verbatim}{commandchars=\\\{\}}
% Add ',fontsize=\small' for more characters per line
\usepackage{framed}
\definecolor{shadecolor}{RGB}{248,248,248}
\newenvironment{Shaded}{\begin{snugshade}}{\end{snugshade}}
\newcommand{\KeywordTok}[1]{\textcolor[rgb]{0.13,0.29,0.53}{\textbf{#1}}}
\newcommand{\DataTypeTok}[1]{\textcolor[rgb]{0.13,0.29,0.53}{#1}}
\newcommand{\DecValTok}[1]{\textcolor[rgb]{0.00,0.00,0.81}{#1}}
\newcommand{\BaseNTok}[1]{\textcolor[rgb]{0.00,0.00,0.81}{#1}}
\newcommand{\FloatTok}[1]{\textcolor[rgb]{0.00,0.00,0.81}{#1}}
\newcommand{\ConstantTok}[1]{\textcolor[rgb]{0.00,0.00,0.00}{#1}}
\newcommand{\CharTok}[1]{\textcolor[rgb]{0.31,0.60,0.02}{#1}}
\newcommand{\SpecialCharTok}[1]{\textcolor[rgb]{0.00,0.00,0.00}{#1}}
\newcommand{\StringTok}[1]{\textcolor[rgb]{0.31,0.60,0.02}{#1}}
\newcommand{\VerbatimStringTok}[1]{\textcolor[rgb]{0.31,0.60,0.02}{#1}}
\newcommand{\SpecialStringTok}[1]{\textcolor[rgb]{0.31,0.60,0.02}{#1}}
\newcommand{\ImportTok}[1]{#1}
\newcommand{\CommentTok}[1]{\textcolor[rgb]{0.56,0.35,0.01}{\textit{#1}}}
\newcommand{\DocumentationTok}[1]{\textcolor[rgb]{0.56,0.35,0.01}{\textbf{\textit{#1}}}}
\newcommand{\AnnotationTok}[1]{\textcolor[rgb]{0.56,0.35,0.01}{\textbf{\textit{#1}}}}
\newcommand{\CommentVarTok}[1]{\textcolor[rgb]{0.56,0.35,0.01}{\textbf{\textit{#1}}}}
\newcommand{\OtherTok}[1]{\textcolor[rgb]{0.56,0.35,0.01}{#1}}
\newcommand{\FunctionTok}[1]{\textcolor[rgb]{0.00,0.00,0.00}{#1}}
\newcommand{\VariableTok}[1]{\textcolor[rgb]{0.00,0.00,0.00}{#1}}
\newcommand{\ControlFlowTok}[1]{\textcolor[rgb]{0.13,0.29,0.53}{\textbf{#1}}}
\newcommand{\OperatorTok}[1]{\textcolor[rgb]{0.81,0.36,0.00}{\textbf{#1}}}
\newcommand{\BuiltInTok}[1]{#1}
\newcommand{\ExtensionTok}[1]{#1}
\newcommand{\PreprocessorTok}[1]{\textcolor[rgb]{0.56,0.35,0.01}{\textit{#1}}}
\newcommand{\AttributeTok}[1]{\textcolor[rgb]{0.77,0.63,0.00}{#1}}
\newcommand{\RegionMarkerTok}[1]{#1}
\newcommand{\InformationTok}[1]{\textcolor[rgb]{0.56,0.35,0.01}{\textbf{\textit{#1}}}}
\newcommand{\WarningTok}[1]{\textcolor[rgb]{0.56,0.35,0.01}{\textbf{\textit{#1}}}}
\newcommand{\AlertTok}[1]{\textcolor[rgb]{0.94,0.16,0.16}{#1}}
\newcommand{\ErrorTok}[1]{\textcolor[rgb]{0.64,0.00,0.00}{\textbf{#1}}}
\newcommand{\NormalTok}[1]{#1}
\usepackage{graphicx,grffile}
\makeatletter
\def\maxwidth{\ifdim\Gin@nat@width>\linewidth\linewidth\else\Gin@nat@width\fi}
\def\maxheight{\ifdim\Gin@nat@height>\textheight\textheight\else\Gin@nat@height\fi}
\makeatother
% Scale images if necessary, so that they will not overflow the page
% margins by default, and it is still possible to overwrite the defaults
% using explicit options in \includegraphics[width, height, ...]{}
\setkeys{Gin}{width=\maxwidth,height=\maxheight,keepaspectratio}
\IfFileExists{parskip.sty}{%
\usepackage{parskip}
}{% else
\setlength{\parindent}{0pt}
\setlength{\parskip}{6pt plus 2pt minus 1pt}
}
\setlength{\emergencystretch}{3em}  % prevent overfull lines
\providecommand{\tightlist}{%
  \setlength{\itemsep}{0pt}\setlength{\parskip}{0pt}}
\setcounter{secnumdepth}{0}
% Redefines (sub)paragraphs to behave more like sections
\ifx\paragraph\undefined\else
\let\oldparagraph\paragraph
\renewcommand{\paragraph}[1]{\oldparagraph{#1}\mbox{}}
\fi
\ifx\subparagraph\undefined\else
\let\oldsubparagraph\subparagraph
\renewcommand{\subparagraph}[1]{\oldsubparagraph{#1}\mbox{}}
\fi

%%% Use protect on footnotes to avoid problems with footnotes in titles
\let\rmarkdownfootnote\footnote%
\def\footnote{\protect\rmarkdownfootnote}

%%% Change title format to be more compact
\usepackage{titling}

% Create subtitle command for use in maketitle
\newcommand{\subtitle}[1]{
  \posttitle{
    \begin{center}\large#1\end{center}
    }
}

\setlength{\droptitle}{-2em}

  \title{MachineProblemRMD}
    \pretitle{\vspace{\droptitle}\centering\huge}
  \posttitle{\par}
    \author{Helen Mary Labao-Barrameda}
    \preauthor{\centering\large\emph}
  \postauthor{\par}
      \predate{\centering\large\emph}
  \postdate{\par}
    \date{11/13/2018}


\begin{document}
\maketitle

\section{Working 6 Problems}\label{working-6-problems}

\subsection{1-WalaNA - The Function that Removes the NA Values in a
Vector}\label{walana---the-function-that-removes-the-na-values-in-a-vector}

\begin{Shaded}
\begin{Highlighting}[]
\NormalTok{walana <-}\StringTok{ }\ControlFlowTok{function}\NormalTok{(a) \{}
  \KeywordTok{return}\NormalTok{(a[}\OperatorTok{!}\KeywordTok{is.na}\NormalTok{(a)])\}}
\NormalTok{testdata <-}\StringTok{ }\KeywordTok{c}\NormalTok{(}\OtherTok{NA}\NormalTok{,}\DecValTok{7}\NormalTok{,}\DecValTok{22}\NormalTok{,}\DecValTok{3}\NormalTok{,}\DecValTok{2}\NormalTok{,}\DecValTok{15}\NormalTok{,}\DecValTok{67}\NormalTok{,}\DecValTok{9}\NormalTok{,}\OtherTok{NA}\NormalTok{,}\OtherTok{NA}\NormalTok{)}
\KeywordTok{walana}\NormalTok{(testdata)}
\end{Highlighting}
\end{Shaded}

\begin{verbatim}
## [1]  7 22  3  2 15 67  9
\end{verbatim}

\begin{Shaded}
\begin{Highlighting}[]
\KeywordTok{print}\NormalTok{(testdata)}
\end{Highlighting}
\end{Shaded}

\begin{verbatim}
##  [1] NA  7 22  3  2 15 67  9 NA NA
\end{verbatim}

\subsection{2 - Recursive Factorial
Function}\label{recursive-factorial-function}

\begin{verbatim}
inputtest<-as.numeric(readline(prompt="Gimme your factorial!"))

FactorialFun <- function(x) {
  if (x==1) {  
    x=1  
  } else {      
    return(x*FactorialFun(x-1));
  }
}

print(FactorialFun(inputtest))
\end{verbatim}

\subsection{3 - Fly Like a POSIX Day
Predictor}\label{fly-like-a-posix-day-predictor}

\begin{verbatim}
#Define an R function that accepts a Date (POSIXct) as argument and outputs the day of the week as characters. Use modulo operator. -- WORKING
FlyLikeaPosix <-as.numeric(readline(prompt="I am the oracle that tells you the day of the POSIXct. Enter the number here."))

# Origin is a Thursday - Jan 1 1970

PosixDayPredictor <- function(x) {
  days<-x/86400
  print(days)
  dayslang <- trunc(days)
  newx <- as.POSIXct(x,tz="UTC", origin = '1970-01-01')
  if(x<86400) {
    day<-"Thursday"
  } else if (x>86400&&x<172800) {
    day<-"Friday"
  } else if(x>604800&&dayslang%%8==0) {
    day<- "Friday"
  } else if(dayslang%%7==0){
    day<-"Thursday"
  } else if(dayslang%%6==0) {
    day<-"Wednesday"
  } else if(dayslang%%5==0) {
    day<-"Tuesday"
  } else if(dayslang%%4==0) {
    day<-"Monday"
  } else if(dayslang%%3==0) {
    day<-"Sunday"
  } else if(dayslang%%2==0) {
    day<-"Saturday"
  }
  print(day)
  print(newx)
  return(day)
}

PosixDayPredictor(FlyLikeaPosix)
\end{verbatim}

\subsection{4 - The Monthly Net Pay
Calculator}\label{the-monthly-net-pay-calculator}

\begin{verbatim}
monthlygross=50000
SalaryCalculatorv2 <- function (basic.monthly, allowance.nontax = 0, allowance.tax = 0) {
    annual.taxable = (basic.monthly + allowance.tax) * 12
    if (annual.taxable <= 250000) {tax = 0}
    else if (annual.taxable <= 400000) {tax = (annual.taxable - 250000) * 0.2} 
    else if (annual.taxable <= 800000) {tax = 30000 + (annual.taxable - 400000) * 0.25}
    else if (annual.taxable <= 2000000) {tax = 130000 + (annual.taxable - 800000) * 0.3}
    else if (annual.taxable <= 8000000) {tax = 490000 + (annual.taxable - 2000000) *0.32}
    else {tax = 2410000 + (annual.taxable - 8000000) * 0.35}
    
    net.monthly = basic.monthly + allowance.nontax + allowance.tax - (tax/12)
    
    return(net.monthly)
  }

SalaryCalculatorv2(monthlygross)
\end{verbatim}

\subsection{5 - Compound Interest
Calculator}\label{compound-interest-calculator}

\begin{verbatim}
#Create a function that computes the compound interest of an investment given the rate, time, and initial amount or principal. -- WORKING
Rate <-as.numeric(readline(prompt="Nominal interest rate in decimal"))
Time <-as.numeric(readline(prompt="Time of investment in years"))
Compounds <-as.numeric(readline(prompt="Compounding periods per year"))
Principal<-as.numeric(readline(prompt="Principal amount"))

CompoundInterest <- function(x,y,z,w) {
  Acompound <- w*(1+(x/z))^(y*z)
  cat("Accrued amount or compounded amount is ", Acompound)
  return(Acompound)
}

CompoundInterest(Rate,Time,Compounds,Principal)
\end{verbatim}

\subsection{6 - Nth Highest Number of a
Vector}\label{nth-highest-number-of-a-vector}

\begin{verbatim}
#Create a function that accepts a vector and and integer n and returns nth highest number
numericvector <- c(3,33,25,46,12,8,9,1,2)

NthHighestNumber <- function(x,y) {
  x<-sort(x,decreasing=TRUE)
  element<-x[y]
  return(element)
}

#Test
NthHighestNumber(numericvector,2)
\end{verbatim}

\section{Coding Still In-Progress}\label{coding-still-in-progress}

\subsubsection{Don't have much time because work and life happens.
=(}\label{dont-have-much-time-because-work-and-life-happens.}

\subsection{SortThingThing -- Bubble Sort for Character/Numeric
Vector}\label{sortthingthing-bubble-sort-for-characternumeric-vector}

\begin{verbatim}
#Define an R function that sorts a given vector in decreasing order. The output should be a vector of the same length. It should accept both numeric or character vectors.
#Work in Progress
#Provided given vectors
numericvector <- c(3,33,25,46,12,8,9,1,2)
charactervector <- c("Luke Skywalker","Han Solo", "Chewbacca", "Darth Vader", "Princess Leia", "Obi Wan Kenobi")

#Bubble Sort for Numeric Vector
SortThisThing <- function(x){
  n<-length(x)
  for(j in 1:(n-1)){
    for(i in 1:(n-j)){
      if(x[i]<x[i+1]){
        temp<-x[i]
        x[i]<-x[i+1]
        x[i+1]<-temp
      }
    }
  }
  return(x)
}

#Working
numberoutput <- SortThisThing(numericvector)
print(numberoutput)

#Not working yet =( 
charoutput <- SortThisThing(charactervector)
print(charactervector)
\end{verbatim}

\subsection{Prime Number Detector}\label{prime-number-detector}

\begin{verbatim}
numero <-as.integer(readline(prompt="Prime or Not? Enter the integer."))

isPrime <- function(n) {
  m<-ceiling(sqrt(n))
  if(n==1) {return(FALSE)}
  else if(n==2|n==3) {return(TRUE)}
  else if(n>3) {
    while(!is.integer(n/m)&&m>=2){
      saywhat<-
      print(saywhat)
      return(saywhat)
      m=m-1
    }
  }
}

isPrime(numero)
\end{verbatim}

\subsection{Determinant Determination}\label{determinant-determination}

\subsubsection{Works for 2x2 and 3x3 already. Still figuring out
recursion for
cofactors.}\label{works-for-2x2-and-3x3-already.-still-figuring-out-recursion-for-cofactors.}

\begin{verbatim}
#Define an R function that computes the determinant of a given matrix. The output should be a vector of length 1.
#Work in Progress
#Test variables so I can check my computation
testdata <- matrix(nrow = 3, ncol = 3)
testdata

testdata[1,1]=0
testdata[1,2]=2
testdata[1,3]=1
testdata[2,1]=3
testdata[2,2]=-1
testdata[2,3]=2
testdata[3,1]=4
testdata[3,2]=-4
testdata[3,3]=1
print(det(testdata))

testdata2 <- matrix(nrow=4,ncol=4)
testdata2[1,1]=0
testdata2[1,2]=2
testdata2[1,3]=1
testdata2[1,4]=3
testdata2[2,1]=-1
testdata2[2,2]=2
testdata2[2,3]=-1
testdata2[2,4]=2
testdata2[3,1]=4
testdata2[3,2]=-4
testdata2[3,3]=1
testdata2[3,4]=0
testdata2[4,1]=34
testdata2[4,2]=1
testdata2[4,3]=4
testdata2[4,4]=-1
det(testdata2)


#Get the dimension of the matrix first
SqMtrxDmnsn <-as.numeric(readline(prompt="Dimension of square matrix"))
inputmatrix <- matrix(0,nrow=SqMtrxDmnsn,ncol=SqMtrxDmnsn)

#Initializer that prompts user to enter the elements of the matrix
#Matrix Writer Function for input matrix
for(i in 1:SqMtrxDmnsn)
  {for(j in 1:SqMtrxDmnsn)
  {
  inputmatrix[i,j] <- as.numeric(readline(prompt="Enter element (L to R then Top to Bottom)"))  
  j=j+1
  }
  i=i+1
  }

detrmntdos <- function(x) {
  #Basic determinant formula
  dos<-x[1,1]*x[2,2]-x[1,2]*x[2,1]
  return(dos)
} 

detrmnttres <- function(x) {
  #Basketweave Method formula for a 3x3 Matrix: 
  tres <- x[1,1]*x[2,2]*x[3,3]+x[1,2]*x[2,3]*x[3,1]+x[1,3]*x[2,1]*x[3,2]-x[3,1]*x[2,2]*x[1,3]-x[3,2]*x[2,3]*x[1,1]-x[3,3]*x[2,1]*x[1,2]
  return(tres)
}

#Matrix by Cofactors Loop Method (I will always use row 1 for expansion to simplify the algo)
#If I still have time: Scale down until it's a 3x3 matrix to apply detrmnttres function from above and return the ultimate determinant
#If Murphy's law at work: I'll use det for matrices higher than 3x3 but improve this code before end of term. 
detrmntgeneral <- function(x) {
  totalna=0
  col=1
  cofactorsum=0
  while(col<=ncol(x)) {
    cofactorthing<-x[-1,(-1*col)]
    print(det(cofactorthing))
    multiplier<-(-1)^(1+col)
    element<-x[1,col]
    print(multiplier)
    print(element)
    print(element*multiplier*det(cofactorthing))
    yup <- element*multiplier*det(cofactorthing)
    cofactorsum = cofactorsum + yup
    print(cofactorsum)
    col=col+1
  }
  return(cofactorsum)
  
}

detrmntgeneral(testdata2)


if (SqMtrxDmnsn==2) {
  detrmntdos(inputmatrix)
} else if (SqMtrxDmnsn==3) {
  detrmnttres(inputmatrix)
} else if (SqMtrxDmnsn>3) {
  detrmntgeneral(inputmatrix)
} else print("Hey, invalid matrix yan, buddy!")
\end{verbatim}


\end{document}
